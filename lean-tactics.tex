\documentclass[a4paper]{article}
\usepackage[T1]{fontenc}
\usepackage[utf8]{inputenc}
\usepackage[english]{babel}

%\usepackage[urw-garamond,expert,uppercase=upright,greeklowercase=upright]{mathdesign}
%\usepackage[osf,swashQ]{garamondx}
%
%\usepackage{microtype}

\usepackage[textwidth=17cm, textheight=27cm]{geometry}
\usepackage{booktabs, xspace}
\usepackage{amsmath,amsfonts,amssymb}

\newcommand{\lean}[1]{{\tt #1}}
\newcommand{\nv}{\textit{new\_name} }
\newcommand{\nom}{\textit{name} }
\newcommand{\expr}{\textit{expr} }
\newcommand{\proposition}{\textit{proposition} }
\newcommand{\hyp}{\textit{hyp}\xspace}

% Original authors (Lean 3 version): Patrick Massot and Johan Commelin
% https://github.com/leanprover-community/lftcm2020/blob/master/lean-tactics.tex

\usepackage{makecell}
\usepackage{xcolor}

\begin{document}
\pagestyle{empty}
\begin{center}
 \large\textsc{Lean 4 Cheatsheet}
\end{center}

In the following tables,
\nom always refers to a name already known to Lean
while \nv is a new name provided by the user;
\expr means an expression,
for example the name of an object in the context,
an arithmetic expression that is a function of such objects,
a hypothesis in the context,
or a lemma applied to any of these;
\proposition is an expression of the type \lean{Prop} (e.g. \lean{0 < x})
When one of these words appears twice in the same cell,
the appearances do not designate the same name or
the same expression.

\begin{center}
\setlength\tabcolsep{5mm}
\def\arraystretch{1.3}
\begin{tabular}{@{}lll@{}}
  \toprule
  Logical symbol & Appears in goal & Appears in hypothesis \\
  \midrule
 $\exists$ (there exists) & \makecell[lt]{\lean{use} \expr} & \makecell[lt]{
\lean{obtain} $\langle$\nv, \nv$\!\!\rangle$ \lean{:=} \expr
} \\
 $\forall$ (for all) & \lean{intro} \nv & \lean{apply} \expr or \lean{specialize} \nom \expr  \\
 $\lnot$ (not) & \lean{intro} \nv & \lean{apply} \expr or \lean{specialize} \nom \expr  \\
 $\to$ (implies) & \lean{intro} \nv & \lean{apply} \expr or \lean{specialize} \nom \expr \\
 $\leftrightarrow$ (if and only if)~~ & \lean{constructor}  & \lean{rw [}\expr\lean{]} or \lean{rw [←} \expr\lean{]}\\
 $\land$ (and) & \lean{constructor} & \makecell[lt]{
\lean{obtain} $\langle$\nv, \nv$\!\!\rangle$ \lean{:=} \expr
} \\
 $\lor$ (or) & \makecell[lt]{\lean{left} or \lean{right}} & \makecell[lt]{
\lean{cases} \expr \lean{with} \\
\lean{| inl }\nv \lean{=> ...} \\
\lean{| inr }\nv \lean{=> ...}
} \\
  \bottomrule
\end{tabular}
\end{center}

\medskip
In the left-hand column of the following table,
the parts in parentheses are optional.
The effect of these parts is also in parentheses in the right-hand column.

\begin{center}
\setlength\tabcolsep{5mm}
\def\arraystretch{1.3}
\begin{tabular}{@{}lp{10cm}@{}}
  \toprule
  Tactic & Effect \\
  \midrule
  \lean{exact} \expr & the goal is satisfied by \expr \\
  \lean{refine} \expr & similar to \lean{exact} but allows to leave any number of \lean{?\_} in the \expr to denote holes that will be filled later (creates goals) \\
  \makecell[lt]{\lean{convert} \expr} & prove the goal by transforming it to an existing fact \expr and create goals for propositions used in the transformation that were not proved automatically \\
  \makecell[lt]{\lean{convert\_to} \proposition} & transform the goal into the goal \proposition and create additional goals for propositions used in the transformation that were not proved automatically \\
  \makecell[lt]{\lean{have} \nv : \proposition} & introduce a name \nv asserting that \proposition is true; at the same time, create and focus a goal for \proposition \\
  \lean{unfold} \nom (\lean{at} \hyp) & unfold the definition \nom in the goal
  (or in the hypothesis \hyp) \\
  \lean{rw [} (\lean{←}) \expr\lean{]} (\lean{at} \hyp) & in the goal (or in the
  hypothesis \hyp), replace (all occurrences of) the left-hand side
  (or the right-hand side, if \lean{←} is present)
  of the equality or equivalence \expr by its other side \\
  \lean{rw [} \expr\lean{,} \expr\lean{,} \expr\lean{]} (\lean{at} \hyp) & do more rewrites in the given order (any number of \lean{←} possible) \\
  \lean{calc} & start a proof by calculation (uses transitivity) \\
  \lean{by\_cases} \nv : \proposition & split the proof into two cases
  depending on whether \proposition is true or false,
  using \nv as name for this hypothesis \\
  \makecell[lt]{\lean{exfalso}} & apply the rule ``False implies anything'' a.k.a.~``ex falso quodlibet'' (replaces the current goal by \lean{False}) \\
  \makecell[lt]{\lean{by\_contra} \nv} & start a proof by contradiction,
  using \nv as name for the hypothesis that is the negation of the goal \\
  \makecell[lt]{\lean{push\_neg} (\lean{at} \hyp)} & push negations in the goal
  (or in the hypothesis \hyp); e.g.~change $\neg\;\forall$ x, \proposition to $\exists$ x, $\neg\;$\proposition \\
  \makecell[lt]{\lean{linarith}} & prove the goal by a linear combination of hypotheses \\
  \makecell[lt]{\lean{ring}} & prove the goal by combining the axioms of a (semi)ring \\
  \makecell[lt]{\lean{simp} (\lean{at} \hyp)} & simplify the goal (or the hypothesis \hyp) using standard equalities \\
  \makecell[lt]{\lean{exact?}} & search for a single existing lemma which closes the goal, also using local hypotheses \\
  \makecell[lt]{\lean{apply?}} & search for lemmas whose conclusion matches the goal; suggest those that may be used with \lean{apply} or \lean{refine} \\
  \makecell[lt]{\lean{aesop}} & try to solve the goal using magic \\
  \bottomrule
\end{tabular}
\end{center}

\end{document}
